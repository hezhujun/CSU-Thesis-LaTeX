\documentclass{csuthesis}

% 设置并排放置多图
\usepackage{subfigure}

\titlecn{零一二三四五六七八九十零一二三四五六七八九十}
\titleen{One Two Three Four Five Six Seven Eight Nine Ten One Two Three Four Five Six Seven Eight Nine Ten}
\author{作者}

\priormajor{计算机科学与技术}
\minormajor{计算机应用技术}
\interestmajor{旁门左道}

\supervisor{xxx\ 教授}
\subsupervisor{}
\department{计算机学院}
\studentid{123456789}

\clcnumber{TP391} 				% 中图分类号 Chinese Library Classification
\schoolcode{10533}			% 学校代码
\udc{004.9}						% UDC
%\academiccategory{学术学位}	 % 学术类别
\academiccategorytrue      % 学术学位,如果是,注释下面一行
%\academiccategoryfalse    % 专业学位,如果是,注释上面一行
%\blindreviewtrue
\blindreviewfalse

% 添加参考文献
% 所有cite格式引用数据放在references.bib文件中
\addbibresource{references.bib}

\begin{document}
    \frontfrontmatter
	%%%%%%%%%%%%%%%%%%%%%%%%%%%%%%%%%%%%%%%%%%%%%%%%%%
	% 封面
	% -----------------------------------------------%
	\makecoverpage
	
	\ifblindreview	% 盲审不需要扉页和声明页
	\else
	%%%%%%%%%%%%%%%%%%%%%%%%%%%%%%%%%%%%%%%%%%%%%%%%%%
	% 扉页 
	% -----------------------------------------------%
	\maketitlepage
	
	%%%%%%%%%%%%%%%%%%%%%%%%%%%%%%%%%%%%%%%%%%%%%%%%%%
	% 声明页
	% -----------------------------------------------%
	\announcement
	\fi

    \cleardoublepage % 重要,必须有这条命令
    \frontmatter
    \begin{abstractcn}
        \noindent
        {\heiti \textbf{摘要:}}摘要摘要摘要摘要摘要摘要摘要摘要摘要摘要摘要摘要摘要摘要摘要摘要摘要摘要摘要摘要摘要摘要摘要摘要摘要摘要摘要摘要摘要摘要摘要摘要摘要摘要摘要摘要摘要摘要摘要摘要摘要摘要摘要摘要摘要摘要摘要摘要摘要摘要摘要摘要摘要摘要摘要摘要。\fontinfo{摘要}。
        
        摘要摘要摘要摘要摘要摘要摘要摘要摘要摘要摘要摘要摘要摘要摘要摘要摘要摘要摘要摘要摘要摘要摘要摘要摘要摘要摘要摘要摘要摘要摘要摘要摘要摘要摘要摘要摘要摘要摘要摘要摘要摘要摘要摘要摘要摘要摘要摘要摘要摘要摘要摘要摘要摘要摘要摘要。
        
        % 空一行
        \vspace*{\baselineskip}
        \noindent
        图 23 幅,表 15 个,参考文献 106 篇
        
		% 空两行
		\vspace*{2\baselineskip}
		\noindent
        {\heiti\textbf{关键字:}}关键词1;关键词2;关键词3\\
        {\heiti\textbf{分类号:}}TP391

    \end{abstractcn}

	\begin{abstracten}
		\noindent
		\textbf{Abstract:} Abstract abstract abstract abstract abstract abstract abstract abstract abstract abstract abstract abstract abstract abstract abstract abstract abstract abstract abstract abstract abstract abstract abstract abstract abstract abstract abstract abstract. \fontinfo{Abstract}
		
		Abstract abstract abstract abstract abstract abstract abstract abstract abstract abstract abstract abstract abstract abstract abstract abstract abstract abstract abstract abstract abstract abstract abstract abstract abstract abstract abstract abstract.
		
		% 空两行
		\vspace*{2\baselineskip}
		\noindent
		{\textbf{Keywords:}} Keywords1; Keywords2; Keywords3\\
		{\textbf{Classification:}} TP391
	\end{abstracten}

    \tableofcontents
    
    \cleardoublepage
    \mainmatter
	\renewcommand{\chaptermark}[1]{\markboth{\thechapter \hspace{0.8em}#1}{}}  % 必须的命令
    \chapter{绪论}
    \thispagestyle{mainstyle} % \chapter后面必须加这条命令启动页眉页脚设置
    \section{课题研究的背景与意义}

    课题研究的背景与意义。课题研究的背景与意义。课题研究的背景与意义。课题研究的背景与意义。课题研究的背景与意义。课题研究的背景与意义。

    课题研究的背景与意义。课题研究的背景与意义。课题研究的背景与意义。课题研究的背景与意义。课题研究的背景与意义。课题研究的背景与意义。

    课题研究的背景与意义。课题研究的背景与意义。课题研究的背景与意义。课题研究的背景与意义。课题研究的背景与意义。课题研究的背景与意义。行距:\the\baselineskip。
    
    \section{ctex字体研究}
    
    {\noindent \zihao{0} \fontinfo{初号}。}
    
    {\noindent \zihao{-0} \fontinfo{小初号}。}
    
    {\noindent \zihao{1} \fontinfo{一号}。}
    
    {\noindent \zihao{-1} \fontinfo{小一号}。}
    
    {\noindent \zihao{2} \fontinfo{二号}。}
    
    {\noindent \zihao{-2} \fontinfo{小二号}。}
    
    {\noindent \zihao{3} \fontinfo{三号}。}
    
    {\noindent \zihao{-3} \fontinfo{小三号}。}
    
    {\noindent \zihao{4} \fontinfo{四号}。}
    
    {\noindent \zihao{-4} \fontinfo{小四号}。MMMMMMMMMMMMMMMMMMMMMMMMMMMMMMMMMMM}
    
    \noindent \fontinfo{正文}。MMMMMMMMMMMMMMMMMMMMMMMMMMMMMMMMMMM
    
    {\noindent \zihao{5} \fontinfo{五号}。MMMMMMMMMMMMMMMMMMMMMMMMMMMMMMMMMMM}
    
    {\noindent \zihao{-5} \fontinfo{小五号}。MMMMMMMMMMMMMMMMMMMMMMMMMMMMMMMMMMM}
    
    {\noindent \zihao{6} \fontinfo{六号}。MMMMMMMMMMMMMMMMMMMMMMMMMMMMMMMMMMM}
    
    {\noindent \zihao{-6} \fontinfo{小六号}。MMMMMMMMMMMMMMMMMMMMMMMMMMMMMMMMMMM}
    
    {\noindent \zihao{7} \fontinfo{七号}。MMMMMMMMMMMMMMMMMMMMMMMMMMMMMMMMMMMMMMMMMMMMMMMMMMMMMMMMMMMMMMMMMMMMM}
    
    {\noindent \zihao{8} \fontinfo{八号}。MMMMMMMMMMMMMMMMMMMMMMMMMMMMMMMMMMMMMMMMMMMMMMMMMMMMMMMMMMMMMMMMMMMMM}
    
    \noindent 段间距 \the\parskip
    
    \section{存在的难点与挑战}

    存在的难点与挑战。存在的难点与挑战。存在的难点与挑战。存在的难点与挑战。存在的难点与挑战。存在的难点与挑战。存在的难点与挑战。存在的难点与挑战。

    存在的难点与挑战。存在的难点与挑战。存在的难点与挑战。存在的难点与挑战。存在的难点与挑战。存在的难点与挑战。存在的难点与挑战。存在的难点与挑战。

    \section{国内外研究现状与发展动态}
    \subsection{国内研究}
    国内研究。国内研究。国内研究。国内研究。国内研究。国内研究。国内研究。国内研究。国内研究。国内研究。国内研究。国内研究。国内研究。

    国内研究。国内研究。国内研究。国内研究。国内研究。国内研究。国内研究。

    \subsection{国外研究}
    国外研究。国外研究。国外研究。国外研究。国外研究。国外研究。

    国外研究。国外研究。国外研究。国外研究。国外研究。国外研究。

    \section{本文主要工作与内容安排}

    本文主要工作与内容安排。本文主要工作与内容安排。本文主要工作与内容安排。本文主要工作与内容安排。本文主要工作与内容安排。

    本文主要工作与内容安排。本文主要工作与内容安排。本文主要工作与内容安排。本文主要工作与内容安排。
    
    本文主要工作与内容安排。本文主要工作与内容安排。本文主要工作与内容安排。本文主要工作与内容安排。本文主要工作与内容安排。
    
    本文主要工作与内容安排。本文主要工作与内容安排。本文主要工作与内容安排。本文主要工作与内容安排。
    
    本文主要工作与内容安排。本文主要工作与内容安排。本文主要工作与内容安排。本文主要工作与内容安排。本文主要工作与内容安排。
    
    本文主要工作与内容安排。本文主要工作与内容安排。本文主要工作与内容安排。本文主要工作与内容安排。
    
    本文主要工作与内容安排。本文主要工作与内容安排。本文主要工作与内容安排。本文主要工作与内容安排。本文主要工作与内容安排。
    
    本文主要工作与内容安排。本文主要工作与内容安排。本文主要工作与内容安排。本文主要工作与内容安排。

    \chapter{研究内容1}
    \thispagestyle{mainstyle} % \chapter后面必须加这条命令启动页眉页脚设置
    \section{介绍}
    介绍。介绍。介绍。介绍。介绍。介绍。

    介绍。介绍。介绍。介绍。介绍。介绍。

    介绍。介绍。介绍。介绍。介绍。介绍。

    \section{相关工作}
    相关工作。相关工作。相关工作。相关工作。

    相关工作。相关工作。相关工作。相关工作。

    \section{方法}
    方法。方法。方法。方法。方法。方法。方法。

    方法。方法。方法。方法。方法。方法。方法。

    公式\ref{eq:eq1}。

    \begin{equation}
        a^2 + b^2 = c^2 \label{eq:eq1}
    \end{equation}

    图片\ref{fig:image1}。
    \begin{figure}
        \centering
        \includegraphics[width=0.8\textwidth]{images/image.jpeg}
        \caption{××图}
        \label{fig:image1}
    \end{figure}

    三线表格\ref{tab:table1}。
    \begin{table}
        \centering
        \caption{**表}
        \label{tab:table1}
        \begin{tabular}{c c}
            \toprule[1.5bp]
            列1 & 列2 \\
            \midrule[0.75bp]
            值1 & 值2 \\
            值1 & 值2 \\
            \bottomrule[1.5bp]
        \end{tabular}
    \end{table}

    如果表格的内容太多,可以把表格的文字缩小,不到迫不得已不要使用表格的续表形式。长表格的续表形式\ref{tab:long_table1}。如果长表格的多个表格之间插入了后文的内容,灵活地调整表格的长度,使得一个表格占一页。或者灵活地调整浮动体table的位置。或者调整后文(看tex文件)。
    \begin{table}[!t]
        \centering
        \caption{长表格}
        \label{tab:long_table1}
        \begin{tabular}{c c}
            \toprule[1.5bp]
            列1 & 列2 \\
            \midrule[0.75bp]
            值1 & 值2 \\
            值1 & 值2 \\
            值1 & 值2 \\
            值1 & 值2 \\
            值1 & 值2 \\
            值1 & 值2 \\
            值1 & 值2 \\
            值1 & 值2 \\
            值1 & 值2 \\
            值1 & 值2 \\
            值1 & 值2 \\
            值1 & 值2 \\
            值1 & 值2 \\
            值1 & 值2 \\
            值1 & 值2 \\
            值1 & 值2 \\
            值1 & 值2 \\
            值1 & 值2 \\
            值1 & 值2 \\
            值1 & 值2 \\
            值1 & 值2 \\
            值1 & 值2 \\
            值1 & 值2 \\
            值1 & 值2 \\
            值1 & 值2 \\
            值1 & 值2 \\
            值1 & 值2 \\
            值1 & 值2 \\
            值1 & 值2 \\
            \bottomrule[1.5bp]
        \end{tabular}
    \end{table}

    %\clearpage % 取消该行注释,同时注释下表后面的\clearpage,看看效果

    \begin{table}[!t]
        \centering
        \caption*{表\thetable \quad 长表格(续)}
        \begin{tabular}{c c}
            \toprule[1.5bp]
            列1 & 列2 \\
            \midrule[0.75bp]
            值1 & 值2 \\
            值1 & 值2 \\
            值1 & 值2 \\
            值1 & 值2 \\
            值1 & 值2 \\
            值1 & 值2 \\
            值1 & 值2 \\
            值1 & 值2 \\
            值1 & 值2 \\
            值1 & 值2 \\
            值1 & 值2 \\
            值1 & 值2 \\
            值1 & 值2 \\
            值1 & 值2 \\
            值1 & 值2 \\
            值1 & 值2 \\
            值1 & 值2 \\
            值1 & 值2 \\
            值1 & 值2 \\
            值1 & 值2 \\
            \bottomrule[1.5bp]
        \end{tabular}
    \end{table}

    \clearpage % 上面是长表格,之后的内容直接在新页面显示

    \section{实验}
    实验。实验。实验。实验。实验。

    实验。实验。实验。实验。实验。
    
    \chapter{研究内容2}
    \thispagestyle{mainstyle} % \chapter后面必须加这条命令启动页眉页脚设置
    \section{介绍}
    介绍。介绍。介绍。介绍。介绍。介绍。

    介绍。介绍。介绍。介绍。介绍。介绍。

    介绍。介绍。介绍。介绍。介绍。介绍。

    \section{相关工作}
    相关工作。相关工作。相关工作。相关工作。

    相关工作。相关工作。相关工作。相关工作。

    \section{方法}
    方法。方法。方法。方法。方法。方法。方法。

    方法。方法。方法。方法。方法。方法。方法。

    公式\ref{eq:eq2}。

    \begin{equation}
        a^2 + b^2 = c^2 \label{eq:eq2}
    \end{equation}

    图片\ref{fig:image2}。
    \begin{figure}
        \centering
        \includegraphics[width=0.8\textwidth]{images/image.jpeg}
        \caption{××图}
        \label{fig:image2}
    \end{figure}

    表格\ref{tab:table2}。
    \begin{table}
        \centering
        \caption{**表}
        \label{tab:table2}
        \begin{tabular}{c|c}
            \toprule[1.5bp]
            列1 & 列2 \\
            \midrule[0.75bp]
            值1 & 值2 \\
            值1 & 值2 \\
            \bottomrule[1.5bp]
        \end{tabular}
    \end{table}

    \section{实验}
    实验。实验。实验。实验。实验。

    实验。实验。实验。实验。实验。

    \chapter{研究内容3}
    \thispagestyle{mainstyle} % \chapter后面必须加这条命令启动页眉页脚设置
    \section{介绍}
    介绍。介绍。介绍。介绍。介绍。介绍。

    介绍。介绍。介绍。介绍。介绍。介绍。

    介绍。介绍。介绍。介绍。介绍。介绍。

    \section{相关工作}
    相关工作。相关工作。相关工作。相关工作。

    相关工作。相关工作。相关工作。相关工作。

    \section{方法}
    方法。方法。方法。方法。方法。方法。方法。

    方法。方法。方法。方法。方法。方法。方法。

    公式\ref{eq:eq3}。

    \begin{equation}
        a^2 + b^2 = c^2 \label{eq:eq3}
    \end{equation}

    图片\ref{fig:image3}。
    \begin{figure}
        \centering
        \includegraphics[width=0.8\textwidth]{images/image.jpeg}
        \caption{××图}
        \label{fig:image3}
    \end{figure}

    表格\ref{tab:table3}。
    \begin{table}
        \centering
        \caption{**表}
        \label{tab:table3}
        \begin{tabular}{c|c}
            \toprule[1.5bp]
            列1 & 列2 \\
            \midrule[0.75bp]
            值1 & 值2 \\
            值1 & 值2 \\
            \bottomrule[1.5bp]
        \end{tabular}
    \end{table}

    \section{实验}
    实验。实验。实验。实验。实验。

    实验。实验。实验。实验。实验。

    \chapter{总结与展望}
    \thispagestyle{mainstyle} % \chapter后面必须加这条命令启动页眉页脚设置
    \section{工作总结}
    工作总结。工作总结。工作总结。工作总结。

    工作总结。工作总结。工作总结。工作总结。

    \section{工作展望}
    工作展望。工作展望。工作展望。工作展望。

    工作展望。工作展望。工作展望。工作展望。
    
    参考文献引用\cite{lamport1994latex};另一个引用\cite{王夫之1977周易外传}。

    % 参考文献
    \printbibliography

    \appendix{第一个附录}
    如果不是一定要加附录,千万不要加附录。
    \asection{附录子标题}
    如果不是一定要加附录,千万不要加附录。
    \asubsection{附录小标题}
    如果不是一定要加附录,千万不要加附录。

    \appendix{第二个附录}
    如果不是一定要加附录,千万不要加附录。
    \asection{附录子标题}
    如果不是一定要加附录,千万不要加附录。
    \asubsection{附录小标题}
    如果不是一定要加附录,千万不要加附录。

    
    % 攻读学位期间主要研究成果
    \achievementtitle
    
    攻读学位期间主要研究成果。攻读学位期间主要研究成果。攻读学位期间主要研究成果。

    攻读学位期间主要研究成果。攻读学位期间主要研究成果。攻读学位期间主要研究成果。
    
    % 致谢
    \thankstitle
    
    致谢。致谢。致谢。致谢。致谢。

    致谢。致谢。致谢。致谢。致谢。

\end{document}
